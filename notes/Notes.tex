\documentclass[leqno]{article}
\usepackage{verbatim}
\usepackage{array}
\usepackage{listings}
\usepackage{fancyvrb}
\usepackage{enumitem}
\usepackage{subcaption}
\usepackage{caption}

\usepackage[utf8]{inputenc}
\usepackage[T1]{fontenc}
\usepackage{textcomp}
\usepackage{multicol} \usepackage{mathtools}
\usepackage{amsmath}
\usepackage{wrapfig}
\usepackage{amssymb}
\usepackage{amsmath,amsfonts,amssymb,amsthm,epsfig,epstopdf,titling,url,array}
\usepackage{hyperref}
\usepackage{eso-pic}
\usepackage{pgf}
\usepackage{tikz}
\usepackage{tikz-cd}
\usepackage{graphicx}

% figure support
\usepackage{import}
\usepackage{xifthen}
\pdfminorversion=7
\usepackage{pdfpages}
\usepackage{transparent}
\usepackage{xcolor}

% geometry
\usepackage{geometry}
\geometry{a4paper, margin=1in}

% paragraph length
\setlength{\parindent}{0em}
\setlength{\parskip}{1em}

\newtheorem*{theorem}{Theorem}
\newtheorem*{lemma}{Lemma}
\newtheorem*{proposition}{Proposition}
\newtheorem*{definition}{Definition}
\newtheorem*{observation}{Observation}

\newcommand{\incfig}[1]{%
\center
\def\svgwidth{0.9\columnwidth}
\import{./figures/}{#1.pdf_tex}
}
\newcommand{\incimg}[1]{%
\center
\includegraphics[width=0.9\columnwidth]{images/#1}
}
\pdfsuppresswarningpagegroup=1

\title{Limit of the covering method for hexagonal lattice}
\author{Abel Doñate Muñoz}
\date{}

\begin{document}
\maketitle
\tableofcontents
\newpage

\section{Basic definitions}
\begin{definition}[Covering lattice]

\end{definition}

\begin{definition}[Limit of covering lattice]

\end{definition}


\begin{figure}[h!]
     \centering
     \begin{subfigure}[b]{0.3\textwidth}
         \centering
         \includegraphics[width=\textwidth]{./figures/lattice2.png}
         \caption{$I=2$. Lattice M}
         \label{fig:I2}
     \end{subfigure}
     \hfill
     \begin{subfigure}[b]{0.3\textwidth}
         \centering
         \includegraphics[width=\textwidth]{./figures/lattice3.png}
         \caption{$I=3$}
         \label{fig:3}
     \end{subfigure}
     \hfill
     \begin{subfigure}[b]{0.3\textwidth}
         \centering
         \includegraphics[width=\textwidth]{./figures/lattice4.png}
         \caption{$I=4$}
         \label{fig:five over x}
     \end{subfigure}
        \caption{Iteration of covering lattices}
        \label{fig:4}
\end{figure}


We can observe the shape of the limit of covering lattice $L$. It reminds us a sort of Sierpinski triangle, but instead of triangles we have the repeating pattern \ref{fig:I2}.

\begin{definition}[Weak vertex] We say a vertex of $L$ is weak if it is one exterior vertex of \ref{fig:I2} 
\end{definition}

\begin{figure}[h!]
    \centering 
    \includegraphics[width=\textwidth]{./figures/sierpinskimod.png}
    \caption{Sierpinski modified triangle}
    \label{fig:sier}
\end{figure}

\begin{observation}All the information we need is encapsulated in the pattern  \ref{fig:I2} and the modification of Sierpinski triangle \ref{fig:sier}.

The lattice $M$ gives us the local information, and the modified Sierpinski the global behaviour setting every weak vertex of $L$ as a vertex of Sierpinski modified triangle.
\end{observation}





\end{document}
