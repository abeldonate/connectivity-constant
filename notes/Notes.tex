\documentclass[leqno]{article}
\usepackage{verbatim}
\usepackage{array}
\usepackage{listings}
\usepackage{fancyvrb}
\usepackage{enumitem}
\usepackage{subcaption}
\usepackage{caption}

\usepackage[utf8]{inputenc}
\usepackage[T1]{fontenc}
\usepackage{textcomp}
\usepackage{multicol} \usepackage{mathtools}
\usepackage{amsmath}
\usepackage{wrapfig}
\usepackage{amssymb}
\usepackage{amsmath,amsfonts,amssymb,amsthm,epsfig,epstopdf,titling,url,array}
\usepackage{hyperref}
\usepackage{eso-pic}
\usepackage{pgf}
\usepackage{tikz}
\usepackage{tikz-cd}
\usepackage{graphicx}
\usepackage{svg}

% figure support
\usepackage{import}
\usepackage{xifthen}
\pdfminorversion=7
\usepackage{pdfpages}
\usepackage{transparent}
\usepackage{xcolor}

% geometry
\usepackage{geometry}
\geometry{a4paper, margin=1in}

% paragraph length
\setlength{\parindent}{0em}
\setlength{\parskip}{1em}

\newtheorem*{theorem}{Theorem}
\newtheorem*{lemma}{Lemma}
\newtheorem*{proposition}{Proposition}
\newtheorem*{definition}{Definition}
\newtheorem*{observation}{Observation}

\newcommand{\incfig}[1]{%
\center
\def\svgwidth{0.9\columnwidth}
\import{./figures/}{#1.pdf_tex}
}
\newcommand{\incimg}[1]{%
\center
\includegraphics[width=0.9\columnwidth]{images/#1}
}
\pdfsuppresswarningpagegroup=1

\title{Limit of the covering method for hexagonal lattice}
\author{Abel Doñate Muñoz}
\date{}

\begin{document}
\maketitle
\tableofcontents
\newpage

\section{Basic definitions}
\begin{definition}[Covering lattice]
Given a lattice $L$ we define its covering lattice $L'$ as follows:
 \begin{enumerate}[topsep=-6pt, itemsep=0pt]
   \item For every vertex of $L$ create a vertex on each of its neighbour edges
   \item Unite the new vertices by an edge (just the ones that are part of the same face)
   \item Unite the vertices that lie in the same edge of $L$
\end{enumerate}
\end{definition}

\begin{definition}[Limit of covering lattice]
The lattice $\mathcal{L}$ is the limit of the coverings $L \to  L' \to L'' \to \ldots$. We will discuss later whether the limit exists.
\end{definition}

If we take the hexagonal lattice and make some iterations the result is the following:
\begin{figure}[h!]
     \centering
     \begin{subfigure}[b]{0.3\textwidth}
         \centering
		 \incfig{lattice2}
         \caption{$I=2$. Lattice M}
         \label{fig:I2}
     \end{subfigure}
     \hfill
     \begin{subfigure}[b]{0.3\textwidth}
         \centering
		 \incfig{lattice3}
         \caption{$I=3$}
         \label{fig:3}
     \end{subfigure}
     \hfill
     \begin{subfigure}[b]{0.3\textwidth}
         \centering
		 \incfig{lattice4}
         \caption{$I=4$}
         \label{fig:five over x}
     \end{subfigure}
        \caption{Iteration of covering lattices}
        \label{fig:4}
\end{figure}


We can observe the shape of the limit of covering lattice $L$. It reminds us a sort of Sierpinski triangle. If we slightly change the geometry we end up with a modification of Sierpinski triangle \ref{fig:sier}


\begin{figure}[h!]
    \centering 
	\incfig{sierpinskimod}
    \caption{Sierpinski modified triangle}
    \label{fig:sier}
\end{figure}




\end{document}
